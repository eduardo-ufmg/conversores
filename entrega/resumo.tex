O trabalho apresenta a determinação e a análise do desempenho de motores de indução trifásicos por meio do circuito equivalente de regime permanente. São obtidos parâmetros elétricos a partir de modelos pré-definidos no Simulink e, com eles, calculam-se grandezas fundamentais no ponto nominal, como escorregamento, velocidade, corrente, torque, potências e eficiência. Em seguida, são geradas curvas características em função do escorregamento e da potência mecânica, permitindo avaliar comportamento em partida, torque máximo e operação a vazio. O estudo inclui ainda a análise da operação em frequência variável, destacando como a variação proporcional tensão-frequência altera as curvas torque-velocidade.