\documentclass[portugues,msc]{ppgccufmg}

\usepackage{lipsum} % remover na versão final

\graphicspath{{../}}

\begin{document}

\ppgccufmg{
    autor={Eduardo Henrique Basilio de Carvalho, Renan Neves da Silva},
    titulo={Trabalho 2 - Motor de Indução},
    cidade={Belo Horizonte},
    ano={2025},
    versao={1.0},
    orientador={Prof. Rodrigo Rodrigues Bastos},
    resumo={resumo.tex},
    abstracten={abstract.tex},
    palavraschave={conversores, motor de indução},
    keywords={converters, induction motor}
}

\chapter{Parâmetros do Circuito Equivalente}

\ref{fig:parametros_preset6} mostra os parâmetros do modelo predefinido de número 6 do bloco \textit{Asynchronous Machine SI Units} do Simulink, para tipo de rotor \textit{Squirrel Cage}.

\begin{figure}[ht]
    \centering
    \caption{Parâmetros do modelo predefinido 6 do bloco \textit{Asynchronous Machine SI Units}.}
    \label{fig:parametros_preset6}
    \includegraphics[width=0.7\textwidth]{T2_M6_params.png} \\
    \small Fonte: captura de tela do programa Simulink na versão R2024b.
\end{figure}

Tais parâmetros são transcritos em \ref{tab:parametros_preset6}.

\begin{table}[ht]
    \centering
    \caption{Parâmetros do modelo predefinido 6 do bloco \textit{Asynchronous Machine SI Units}.}
    \label{tab:parametros_preset6}
    \begin{tabular}{lll}
        \hline
        \textbf{Parâmetro} & \textbf{Símbolo} & \textbf{Valor} \\ \hline
        Potência nominal & $P_{n}$ & 111.9 kVA \\
        Tensão nominal de linha & $V_{n}$ & 460 V\textsubscript{rms} \\
        Frequência nominal & $f_{n}$ & 60 Hz \\
        Resistência do estator & $R_{s}$ & 0.0302 $\Omega$ \\
        Indutância do estator & $L_{s}$ & 0.000283 H \\
        Resistência do rotor & $R_{r}'$ & 0.1721 $\Omega$ \\
        Indutância do rotor & $L_{r}'$ & 0.000283 H \\
        Indutância mútua & $L_{m}$ & 0.01095 H \\
        Inércia & $J$ & 2 kg.m$^{2}$ \\
        Fator de fricção & $F$ & 0.02154 Nms \\
        Número de pares de polos & $p$ & 2 \\ \hline
    \end{tabular}
\end{table}

\end{document}