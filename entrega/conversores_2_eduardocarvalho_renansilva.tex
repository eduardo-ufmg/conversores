\documentclass[portugues,msc]{ppgccufmg}
\usepackage[brazil]{babel}
\usepackage{graphicx}
\usepackage{amsmath}
\usepackage{cleveref}

\graphicspath{{../}}

\newcommand{\crefrangeconjunction}{ -- }
\crefname{figure}{Figura}{Figuras}
\crefname{table}{Tabela}{Tabelas}
\crefname{equation}{Equação}{Equações}

\begin{document}

\ppgccufmg{
    autor={Eduardo Henrique Basilio de Carvalho, Renan Neves da Silva},
    titulo={Trabalho 2 - Motor de Indução},
    cidade={Belo Horizonte},
    ano={2025},
    versao={1.0},
    orientador={Prof. Rodrigo Rodrigues Bastos},
    resumo={resumo.tex},
    abstracten={abstract.tex},
    palavraschave={conversores, motor de indução},
    keywords={converters, induction motor}
}

\chapter{Parâmetros do Circuito Equivalente}

\cref{fig:parametros_preset6} mostra os parâmetros do modelo predefinido de número 6 do bloco \textit{Asynchronous Machine SI Units} do Simulink, para tipo de rotor \textit{Squirrel Cage}.

\begin{figure}[ht]
    \centering
    \caption{Parâmetros do modelo predefinido 6 do bloco \textit{Asynchronous Machine SI Units}.}
    \label{fig:parametros_preset6}
    \includegraphics[width=0.7\textwidth]{T2_M6_params.png} \\
    \small Fonte: captura de tela do programa Simulink na versão R2024b.
\end{figure}

Tais parâmetros são transcritos em \cref{tab:parametros_preset6}.

\begin{table}[ht]
    \centering
    \caption{Parâmetros do modelo predefinido 6 do bloco \textit{Asynchronous Machine SI Units}.}
    \label{tab:parametros_preset6}
    \begin{tabular}{lll}
        \hline
        \textbf{Parâmetro} & \textbf{Símbolo} & \textbf{Valor} \\ \hline
        Potência nominal & $P_{n}$ & 111.9 kVA \\
        Tensão nominal de linha & $V_{n}$ & 460 V\textsubscript{rms} \\
        Frequência nominal & $f_{n}$ & 60 Hz \\
        Resistência do estator & $R_{s}$ & 0.0302 $\Omega$ \\
        Indutância do estator & $L_{s}$ & 0.000283 H \\
        Resistência do rotor & $R_{r}'$ & 0.01721 $\Omega$ \\
        Indutância do rotor & $L_{r}'$ & 0.000283 H \\
        Indutância mútua & $L_{m}$ & 0.01095 H \\
        Inércia & $J$ & 2 kg.m$^{2}$ \\
        Fator de fricção & $F$ & 0.02154 Nms \\
        Número de pares de polos & $p$ & 2 \\ \hline
    \end{tabular}
\end{table}

Destes, os parâmetros do circuito equivalente são calculados em \crefrange{eq:velocidade_angular_eletrica}{eq:resistencia_rotor}.

\begin{align}
    w_{e} & = 2 \pi f_{n} = 2 \pi \cdot 60 = 376.99 \text{ rad/s} \\ \label{eq:velocidade_angular_eletrica}
    V_{ph} & = \frac{V_{n}}{\sqrt{3}} = \frac{460}{\sqrt{3}} = 265.58 \text{ V} \\
    R_{s} & = 0.0302 \Omega \\
    X_{s} & = w_{e} L_{s} = 376.99 \cdot 0.000283 = 0.1067 \Omega \\
    X_{m} & = w_{e} L_{m} = 376.99 \cdot 0.01095 = 4.126 \Omega \\
    X_{r}' & = w_{e} L_{r}' = 376.99 \cdot 0.000283 = 0.1067 \Omega \\
    R_{r} & = \frac{R_{r}'}{s} = \frac{0.01721}{s} \Omega \label{eq:resistencia_rotor}
\end{align}

Este trabalho considera $P_n = P_\text{out}$.

\cref{fig:circuito_equivalente} mostra o circuito equivalente do motor de indução.

\begin{figure}[ht]
    \centering
    \caption{Circuito equivalente do motor de indução.}
    \label{fig:circuito_equivalente}
    \includegraphics[width=0.5\textwidth]{eqv.png} \\
    \small Fonte: captura de tela do programa KiCad na versão 9.0.x.
\end{figure}

\chapter{Operação em Condições Nominais}

\section{Escorregamento}

Do circuito,

\begin{align}
    Z_{r}'(s) &= \frac{R_{r}'}{s} + jX_{r}' \\
    Z_{s} &= R_{s} + jX_{s} \\
    Z_{m} &= jX_{m} \\
    Z_{par}(s) &= \frac{Z_{m}Z_{r}'(s)}{Z_{m} + Z_{r}'(s)} \\
    Z_{eq}(s) &= Z_{s} + Z_{par}(s) \\
    I_{s}(s) &= \frac{V_{ph}}{Z_{eq}(s)} \\
    I_{r}'(s) &= I_{s}(s)\frac{Z_{m}}{Z_{m}+Z_{r}'(s)} \\
    \left|I_{r}'(s)\right| &= \frac{V_{ph}X_{m}}{\sqrt{\text{Re}^2 + \text{Im}^2}} \\
    P_{n} &= 3\left|I_{r}'(s)\right|^{2}R_{r}'\frac{1-s}{s}
\end{align}

para

\begin{align}
    \text{Re} &= \frac{R_{s}R_{r}'}{s} - X_{s}(X_{m} + X_{r}') - X_{m}X_{r}' \\
    \text{Im} &= R_{s}(X_{m} + X_{r}') + \frac{R_{r}'}{s}(X_{s} + X_{m}) \\
\end{align}

No motor estudado,

\begin{align}
    \text{Re} &= \frac{0.0302 \cdot 0.01721}{s} - 0.1067(4.126 + 0.1067) - 4.126 \cdot 0.1067 \\
       &= \frac{0.000519742}{s} - 0.891873 \\
    \text{Im} &= 0.0302(4.126 + 0.1067) + \frac{0.01721}{s}(0.1067 + 4.126) \\
       &= 0.127828 + \frac{0.0728448}{s} \\
    \text{Re}^{2} + \text{Im}^2 &= \left(\frac{0.000519742}{s} - 0.891873\right)^{2} + \left(0.127828 + \frac{0.0728448}{s}\right)^{2} \\
       &= \frac{0.812s^{2} + 0.0177s + 0.00531}{s^{2}} \\
\end{align}

Assim,

\begin{align}
    111900 &= 3 \left(\frac{265.58 \cdot 4.126}{\sqrt{\frac{0.812s^{2} + 0.0177s + 0.00531}{s^{2}}}}\right)^{2}0.01721\frac{1-s}{s} \\
    111900 &= 0.05163 \frac{1200740}{\frac{0.812s^{2} + 0.0177s + 0.00531}{s^{2}}}\frac{1-s}{s} \\
    111900 &= 61994\frac{s(1-s)}{0.812s^{2} + 0.0177s + 0.00531} \\
    1.805 &= \frac{s - s^{2}}{0.812s^{2} + 0.0177s + 0.00531} \\
    s - s^{2} &= 1.466s^{2} + 0.0319s + 0.00959 \\
    0 &= 2.466s^{2} - 0.9681s + 0.00959 \\
\end{align}

Cujas soluções são

\begin{align}
    s_{0} &= 0.0102 \\
    s_{1} &= 0.3824
\end{align}

Conforme \cite{Chapman2013},

\begin{equation}
    s_\text{max} = \frac{R_{r}'}{\sqrt{R_{s}^{2}\left( \frac{X_{m}}{ X_{s} + X_{m} } \right)^{4} + (X_{s} + X_{r}')^{2}}}
\end{equation}

Para os valores considerados,

\begin{align}
    s_\text{max} &= \frac{0.01721}{\sqrt{0.0302^{2}\left( \frac{4.126}{ 0.1067 + 4.126 } \right)^{4} + (0.1067 + 0.1067)^{2}}} \\
       &\approx 0.08
\end{align}

Por operar em condições nominais, $s < s_\text{max}$. Portanto, 

\begin{equation}
    s = s_{0} = 0.0102 = 1.02\%
\end{equation}

\section{Velocidade Mecânica}

\begin{align}
    n_{sinc} &= \frac{60\cdot{}f_{n}}{p} \\
    n_{sinc} &= \frac{60\cdot60}{2} \\
             &= 1800 \text{ rpm} \\
    n &= n_{sinc}(1 - s) \\
      &= 1800(1 - 0.0102) \\
      &= 1781.64 \text{ rpm}
\end{align}

\section{Corrente do Estator}

Sob escorregamento nominal,

\begin{align}
    Z_{r}' &= \frac{0.01721}{0.0102} + j0.1067 \\
           &= 1.687 + j0.1067\ \Omega \\
    Z_{par} &= \frac{j4.126(1.687 + j0.1067)}{j4.126 + 1.687 + j0.1067} \\
           &= 1.384 + j0.655\ \Omega \\
    Z_{eq} &= (0.0302 + j0.1067) + (1.384 + j0.655) \\
           &= 1.414 + j0.762\ \Omega \\
           &= 1.606 \angle 28.3^\circ\ \Omega \\
\end{align}

Então,

\begin{align}
    I_{s} &= \frac{265.58}{1.606 \angle 28.3^\circ} \\
          &= 165.4 \angle -28.3^\circ\ \text{A}
\end{align}

\section{Torque}

Na velocidade mecânica nominal,

\begin{align}
    w_{m} &= \frac{2\pi n}{60} \\
           &= \frac{2\pi \cdot 1781.64}{60} \\
           &= 186.7\ \text{rad/s} \\
    T_{n} &= \frac{P_{n}}{w_{m}} \\
           &= \frac{111900}{186.7} \\
           &= 599.5\ \text{Nm}
\end{align}

\section{Potência Ativa de Entrada}

Pelas grandezas de linha,

\begin{align}
    P_\text{in} &= \sqrt{3} V_{n} I_{s} \cos\phi \\
           &= \sqrt{3} \cdot 460 \cdot 165.4 \cdot \cos 28.3^\circ \\
           &= 116030\ \text{W}
\end{align}

\section{Potência Reativa de Entrada}

Pelas grandezas de linha,

\begin{align}
    Q_\text{in} &= \sqrt{3} V_{n} I_{s} \sin\phi \\
           &= \sqrt{3} \cdot 460 \cdot 165.4 \cdot \sin 28.3^\circ \\
           &= 62476\ \text{var}
\end{align}

\section{Potência Aparente de Entrada}

Por definição,

\begin{align}
    S_\text{in} &= P_\text{in} + jQ_\text{in} \\
           &= 116030 + j62476\ \text{VA} \\
    \left| S_\text{in} \right| &= \sqrt{P_\text{in}^{2} + Q_\text{in}^{2}} \\
           &= \sqrt{116030^{2} + 62476^{2}} \\
           &= 131.8\ \text{kVA}
\end{align}

\section{Fator de Potência}

Por definição,

\begin{align}
    \text{fp} &= \cos\phi \text{ característica }\\
              &= \cos 28.3^\circ \text{ atrasado} \\
              &= 0.880 \text{ atrasado}
\end{align}

\section{Eficiência}

Pelas potências,

\begin{align}
    \eta &= \frac{P_\text{out}}{P_\text{in}} \\
         &= \frac{111900}{116030} \\
         &= 0.9645 \\
         &= 96.45\%
\end{align}

\section{Características Nominais}

\cref{tab:caracteristicas_nominais} apresenta as características nominais calculadas.

\begin{table}[ht]
    \centering
    \caption{Características nominais calculadas.}
    \label{tab:caracteristicas_nominais}
    \begin{tabular}{lll}
        \hline
        \textbf{Grandeza} & \textbf{Símbolo} & \textbf{Valor} \\ \hline
        Escorregamento & $s$ & 1.02\% \\
        Velocidade mecânica & $n$ & 1781.64 rpm \\
        Corrente de linha do estator & $I_s$ & 165.4 A \\
        Torque nominal & $T_n$ & 599.5 Nm \\
        Potência ativa de entrada & $P_\text{in}$ & 116030 W \\
        Potência reativa de entrada & $Q_\text{in}$ & 62476 var \\
        Potência aparente de entrada & $|S_\text{in}|$ & 131.8 kVA \\
        Fator de potência & $\text{fp}$ & 0.880 atrasado \\
        Eficiência & $\eta$ & 96.45\% \\ \hline
    \end{tabular}
\end{table}

\chapter{Curvas}



\bibliographystyle{plain}
\bibliography{referencias}

\end{document}